\documentclass[../main.tex]{subfiles}
\begin{document}
\section{Theoretical setup}\label{sec:setup}

We illustrate the core ideas and principles behind projection-based ROMs. 
Consider a simple toy model \eqref{eq:ibvp} with linear operator $\mathcal{L}_{\mu}$ and (possibly nonlinear) state-dependent source $f$, where $\mu\in\mathcal{P}$ models the parametric dependence of the solution field $u(x,t)$ defined over the $d-$dimensional domain $\Omega$ and over time $T>0$.
We introduce a manifold $\mathcal{M}:=\{u(\mu)\in H^{1}(\Omega)\times\mathbb{R}^{+}\;:\; \mu\in \mathcal{P}\}$ of (weak) solutions of the parametric IBVP \eqref{eq:ibvp} and two, finite-dimensional, subspaces $\mathcal{M}_{h} \subset \mathcal{M}$ for the high-fidelity approximation and $\mathcal{R}_{h} \subset \mathcal{M}_{h}$ for the reduced approximation.
The classical workflow of projection-based ROMs is divided in two main stages: at first a training phases finds high-order numerical solutions in $\mathcal{M}_{h}$ which will populate a sample set for various parameter values; at the end of the training set we construct a reduced-order solution set $\mathcal{R}_{h}$ by projection.

% Offline/training phase 
\subfile{subsec:offline}
% Online/generalisation phase 
\subfile{subsec:online}
% Results 
\subfile{subsec:classical_results}

\end{document}
