\documentclass[../main.tex]{subfiles}
\begin{document}
\subsection{Online/generalisation phase}\label{subsec:online}
Now that we have assembled a ROM we aim at producing new solutions for unseen values of the parameter $\mu\in \mathcal{P}\setminus P$ by exploiting this reduction.
To describe a low-dimensional solution in the ROM we formulate the following ansatz

\begin{equation}\label{eq:ansatz}
        \boldsymbol{u}_{h}(t,\mu)\approx\Pi_{\mathcal{R}_{h}}\boldsymbol{u}_{h} = \sum_{j=1}^{N_{r}}\bar{u}_{j}(t,\mu)\,\boldsymbol{v}_{j} = \boldsymbol{V}_{r}\,\bar{\boldsymbol{u}}\,.
\end{equation}

We can compute a representation of the solution $\boldsymbol{u}_{h}$ in the ROM $\mathcal{R}_{h}$ via \eqref{eq:ansatz} if we can uniquely determine the coefficients $\bar{\boldsymbol{u}}\in \mathbb{R}^{N_{r}}$ of the basis $\text{col}(\boldsymbol{V}_{r})$.

% Projection onto the ROM 
\subfile{subsubsec:projection}
% Kolmogoroff n-width decay 
\subfile{subsubsec:kol_decay}

\end{document}
